\documentclass[a4paper,10pt]{article}
\usepackage[utf8]{inputenc}

\usepackage{amsmath}
\usepackage{amsfonts}
\usepackage{amssymb}


%Para \lstset e insertar codigo
\usepackage{listings}
\usepackage{color}

\lstset{%
  frame=tb,
  language=Octave,%linguagem por defeito
  %
  aboveskip=3mm,
  belowskip=3mm,
  %backgroundcolor=\color{myverylowgray},
  showstringspaces=false,
  columns=flexible,
  basicstyle={\small\ttfamily},
  %
  numbers=none,
  numberstyle=\tiny\color{mygray},
  %
  breaklines=true,
  breakatwhitespace=true,
  tabsize=4
}

%opening
\title{pds\_bersbceo.m}
\author{Fernando Pujaico Rivera}

\begin{document}

\maketitle
%%%%%%%%%%%%%%%%%%%%%%%%%%%%%%%%%%%%%%%%%%%%%%%%%%%%%%%%%%%%%%%%%%%%%%%%%%%%%%%%
\begin{abstract}
The function pds\_bersbceo() represent the formulation of Bit Error Rate ($BER$) 
in the symmetric binary CEO problem.
\end{abstract}

%%%%%%%%%%%%%%%%%%%%%%%%%%%%%%%%%%%%%%%%%%%%%%%%%%%%%%%%%%%%%%%%%%%%%%%%%%%%%%%%
\section{Introduction}
The function pds\_bersbceo() in the m-file pds\_bersbceo.m is defined as:
\lstset{language=octave}%orden importa
\begin{lstlisting}
 E = pds_bersbceo(Ps,M);
\end{lstlisting}
This function represent the formulation of $BER$ in the symmetric binary CEO problem.

\begin{equation}\label{eq:bersbceo}
BER= 
\begin{cases}
\sum \limits_{k=\lfloor \frac{M}{2} \rfloor +1}^{M} \binom{M}{k}  (1-P_s)^{M-k} P_s^k & \text{ if } M~odd \\ 
~ & ~ \\
\sum \limits_{k=\lfloor \frac{M}{2} \rfloor +1}^{M} \binom{M}{k}  (1-P_s)^{M-k} P_s^k & \text{ if } M~even \\
~~~~+0.5~\binom{M}{\frac{M}{2}}  (1-P_s)^{\frac{M}{2}} P_s^{\frac{M}{2}} & ~ \\
\end{cases}
\end{equation}
This form is showed in \cite{berbinaryceo}, a similar form is presented in \cite{Abrardo2009,Ferrari2012}.

\subsection{Working with the probability $BER=P(\hat{U_0} \neq U_0)$}
In \cite{Abrardo2009,Ferrari2012} is considered a maximum a posteriori ($MAP$) 
fusion rule $f(\Omega_M)$, 
where the output value $\hat{u}_0$ of $\hat{U}_0$
is obtained as
\begin{equation}
\begin{matrix}
 \hat{u}_0 & =      & \arg_{u_0} \max P(U_0|\Omega_M)\\
 ~         & \equiv & \arg_{u_0} \max P(\Omega_M|U_0)\
\end{matrix}
\end{equation}
Thus, considering that $m_0$ is the number of zeros in $\Omega_M$, the decision is simplify to 
\begin{equation}
m_0
\begin{matrix}
\hat{u}_0=1 \\
\geq \\
< \\
\hat{u}_0=0
\end{matrix}
\lfloor \frac{M}{2}\rfloor
\end{equation}
In this expression is considered that if  $M$ is even and $m_0=M/2$, the decision 
is arbitrarily assume that $\hat{u}_0 = 1$,
so that  $P(\hat{U_0} \neq U_0)$ is $BER=0.5~[ P(\hat{u}_0=0|u_0=1)+P(\hat{u}_0=1|u_0=0) ]$,
\begin{equation} \label{eq:abrardo}
 BER= 0.5 \sum_{k=0}^{\lfloor \frac{M}{2} \rfloor -1}  \binom{M}{k} {(1-P_s)}^{k}   {P_s}^{M-k} 
    + 0.5 \sum_{k=\lfloor \frac{M}{2}     \rfloor}^{M} \binom{M}{k} {(1-P_s)}^{M-k} {P_s}^{k}   
\end{equation}
where, $\lfloor . \rfloor$ is the floor function and the value $BER$ only is valid for
values of $P_s \leq 1/2$ 
\footnote{Here is important note that in \cite{Abrardo2009,Ferrari2012} your value $\rho$
is equal to $1-P_s$ here, and your result is for $\rho > 0.5$}. The Equation 
(\ref{eq:abrardo}) can be sort as (\ref{eq:bersbceo}).
 
 \begin{thebibliography}{99}
 
\bibitem{berbinaryceo}
Haghighat, J.; Behroozi, Hamid; Plant, D.V., "Iterative joint decoding for sensor 
networks with binary CEO model," Signal Processing Advances in Wireless 
Communications, 2008. SPAWC 2008. IEEE 9th Workshop on , vol., no., pp.41,45, 
6-9 July 2008. doi: 10.1109/SPAWC.2008.4641566
 
\bibitem{Abrardo2009}
Abrardo, A.; Ferrari, G.; Martalò, M.; Perna, F. Feedback Power Control Strategies in 
Wireless Sensor Networks with Joint Channel Decoding. Sensors 2009, 9, 8776-8809.
doi:10.3390/s91108776
 
\bibitem{Ferrari2012}
Ferrari, G.; Martalo, M.; Abrardo, A.; Raheli, R., "Orthogonal multiple access 
and information fusion: How many observations are needed?," Information Theory and 
Applications Workshop (ITA), 2012 , vol., no., pp.311,320, 5-10 Feb. 2012.
doi: 10.1109/ITA.2012.6181783


 \end{thebibliography} 
\end{document}
