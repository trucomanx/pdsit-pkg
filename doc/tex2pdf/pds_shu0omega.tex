\documentclass[a4paper,10pt]{article}
\usepackage[utf8]{inputenc}

\usepackage{amsmath}
\usepackage{amsfonts}
\usepackage{amssymb}


%Para \lstset e insertar codigo
\usepackage{listings}
\usepackage{color}

\lstset{%
  frame=tb,
  language=Octave,%linguagem por defeito
  %
  aboveskip=3mm,
  belowskip=3mm,
  %backgroundcolor=\color{myverylowgray},
  showstringspaces=false,
  columns=flexible,
  basicstyle={\small\ttfamily},
  %
  numbers=none,
  numberstyle=\tiny\color{mygray},
  %
  breaklines=true,
  breakatwhitespace=true,
  tabsize=4
}

%opening
\title{pds\_shu0omega.m}
\author{Fernando Pujaico Rivera}

\begin{document}

\maketitle
%%%%%%%%%%%%%%%%%%%%%%%%%%%%%%%%%%%%%%%%%%%%%%%%%%%%%%%%%%%%%%%%%%%%%%%%%%%%%%%%
\begin{abstract}
The function pds\_shu0omega() represent the formulation of $E = H(U_0|\Omega_M)$ 
in symmetric case of binary CEO problem.
\end{abstract}

%%%%%%%%%%%%%%%%%%%%%%%%%%%%%%%%%%%%%%%%%%%%%%%%%%%%%%%%%%%%%%%%%%%%%%%%%%%%%%%%
\section{Introduction}
The function pds\_shu0omega() in the m-file pds\_shu0omega.m is defined as:
\lstset{language=octave}%orden importa
\begin{lstlisting}
 E = pds_shu0omega(Ps,M).
\end{lstlisting}
This function represent the formulation of $E=H(U_0|\Omega_M)$ in symmetric case 
of binary CEO problem,
being $\Omega_M=U_1 U_2$ $ ...$ $ U_M$, $Pr(U_0)=0.5$ and $P_s = Pr(U_i \neq U_0|U_0)$.
\begin{equation}
 E = \sum_{k=0}^M \binom{M}{k} P_s^k (1-P_s)^{M-k} log_2\left ( 1 +  \left\{\frac{P_s}{(1-P_s)}\right\}^{M-2k} \right )
\end{equation}

This equation can be seen in the thesis \cite{shu0omega} in the page 49.

\begin{thebibliography}{99}
 
\bibitem{shu0omega}
  Heshmati, Ashkan   (2007) Data compression and transmission in Wireless Sensor Networks.  
  Masters thesis, Concordia University. URL http://spectrum.library.concordia.ca/975271/ 

\end{thebibliography}  

\end{document}
