\documentclass[a4paper,10pt]{article}
\usepackage[utf8]{inputenc}

\usepackage{amsmath}
\usepackage{amsfonts}
\usepackage{amssymb}


%Para \lstset e insertar codigo
\usepackage{listings}
\usepackage{color}

\lstset{%
  frame=tb,
  language=Octave,%linguagem por defeito
  %
  aboveskip=3mm,
  belowskip=3mm,
  %backgroundcolor=\color{myverylowgray},
  showstringspaces=false,
  columns=flexible,
  basicstyle={\small\ttfamily},
  %
  numbers=none,
  numberstyle=\tiny\color{mygray},
  %
  breaklines=true,
  breakatwhitespace=true,
  tabsize=4
}

%opening
\title{hb\_joint\_sbceo.m}
\author{Fernando Pujaico Rivera}

\begin{document}

\maketitle
%%%%%%%%%%%%%%%%%%%%%%%%%%%%%%%%%%%%%%%%%%%%%%%%%%%%%%%%%%%%%%%%%%%%%%%%%%%%%%%%
\begin{abstract}
The function hb\_joint\_sbceo() represent the formulation of $E = H(\Omega_M)$ 
in symmetric case of binary CEO problem.
\end{abstract}

%%%%%%%%%%%%%%%%%%%%%%%%%%%%%%%%%%%%%%%%%%%%%%%%%%%%%%%%%%%%%%%%%%%%%%%%%%%%%%%%
\section{Introduction}
The function hb\_joint\_sbceo() in the m-file hb\_joint\_sbceo.m is defined as:
\lstset{language=octave}%orden importa
\begin{lstlisting}
 E = hb_joint_sbceo(Ps,M).
\end{lstlisting}
This function represent the formulation of $E=H(\Omega_M)$ in symmetric case 
of binary CEO problem,
being $\Omega_M=U_1 U_2$ $ ...$ $ U_M$, $Pr(U_0)=0.5$ and $P_s = Pr(U_i \neq U_0|U_0)$.
\begin{equation}
 E = -\sum_{k=0}^M \binom{M}{k} Prob(k)log_2( Prob(k) )
\end{equation}

\begin{equation}
 Prob(k)=0.5 ( Ps^k (1-Ps)^{M-k} + Ps^{M-k} (1-Ps)^k )
\end{equation}


This equation can be seen in \cite{shomega}.

\begin{thebibliography}{99}
 
\bibitem{shomega}
Ferrari, G.; Martalo, M.; Abrardo, A.; Raheli, R., "Orthogonal multiple 
access and information fusion: How many observations are needed?," 
Information Theory and Applications Workshop (ITA), 2012 , vol., no., 
pp.311,320, 5-10 Feb. 2012. doi: 10.1109/ITA.2012.6181783

\end{thebibliography}  

\end{document}
