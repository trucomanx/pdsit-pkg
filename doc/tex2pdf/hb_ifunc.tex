\documentclass[a4paper,10pt]{article}
\usepackage[utf8]{inputenc}

\usepackage{amsmath}
\usepackage{amsfonts}
\usepackage{amssymb}


%Para \lstset e insertar codigo
\usepackage{listings}
\usepackage{color}

\lstset{%
  frame=tb,
  language=Octave,%linguagem por defeito
  %
  aboveskip=3mm,
  belowskip=3mm,
  %backgroundcolor=\color{myverylowgray},
  showstringspaces=false,
  columns=flexible,
  basicstyle={\small\ttfamily},
  %
  numbers=none,
  numberstyle=\tiny\color{mygray},
  %
  breaklines=true,
  breakatwhitespace=true,
  tabsize=4
}

%opening
\title{hb\_ifunc.m}
\author{Fernando Pujaico Rivera}

\begin{document}

\maketitle
%%%%%%%%%%%%%%%%%%%%%%%%%%%%%%%%%%%%%%%%%%%%%%%%%%%%%%%%%%%%%%%%%%%%%%%%%%%%%%%%
\begin{abstract}
The function hb\_ifunc() represent the inverse function of binary entropy.
\end{abstract}

%%%%%%%%%%%%%%%%%%%%%%%%%%%%%%%%%%%%%%%%%%%%%%%%%%%%%%%%%%%%%%%%%%%%%%%%%%%%%%%%
\section{Introduction}
The function hb\_ifunc() in the m-file hb\_ifunc.m is defined as:
\lstset{language=octave}%orden importa
\begin{lstlisting}
 P = hb_ifunc(H);
\end{lstlisting}
This function represent the formulation of inverse function $h_b^{-1}()$ of binary entropy 
function $h_b()$; being  $H$ the binary entropy of probability $P$.
\begin{equation}\label{eq:hbinv}
 P=h_b^{-1}(H)
\end{equation}

\begin{equation}\label{eq:hb}
H=h_b(P)=-P~log_2(P) -(1-P)~log_2(1-P)
\end{equation}
This form is showed in \cite{thomas}.

 \begin{thebibliography}{99}
 
\bibitem{thomas}
Thomas M. Cover and Joy A. Thomas. Elements of Information Theory 2nd Edition. 
Wiley Series in Telecommunications and Signal Processing. 
Wiley-Interscience, 2 edition, July 2006.



 \end{thebibliography} 
\end{document}
